\begin{thebibliography}{99}
 \bibitem{takagi} 高木 貞治,定本 解析概論,岩波書店 \\
理系ならとりあえず持っておけと言われるくらいの名著.古臭い口調で一見読みにくいが,
慣れるとスラスラ読める(これは口調に慣れるという意味で内容がスラスラ入るという意味ではない).
数学をきちっとやりたい人向け.それなりに難易度は高めで,
覚悟を決めて長く付き合う本である.
\bibitem{kodaira} 小平 邦彦,軽装版 解析入門I,岩波書店 \\
こっちも$\varepsilon-\delta$論法を学ぶための本.
\cite{takagi}よりも難易度は低めで読みやすい.IとI\hspace{-.1em}Iの二巻構成.
それにしても「解析入門」ってタイトルの本多すぎませんかねぇ.
\bibitem{kaiseki}和達 三樹,微分積分(理工系の数学入門コース 1),岩波書店 \\
解析系3冊目.おそらくこれが一番読みやすいだろう.
本書の読者層を考えると,私が見てきた中では解析系の本で一番推したい1冊である. 
 \bibitem{liner} 三宅 敏恒,線形代数学-初歩からジョルダン標準形へ,培風館 \\
本書ではゴッソリ省いた線形代数学に関する本.簡潔にまとまっていて読みやすい.
こういう本が読めるようになるとスムーズに学習が進むようになる.
なお,初めの行列計算をやっているときには意味は深く考えなくてもいい.
一歩先に進み,ベクトル空間について学ぶと
行列計算の意味が驚くほど鮮明に見えてくるはずだ.
物理を学ぶ者としてはそこにたどり着くのが最初のゴールといえるだろう.
\bibitem{matsuzaka} 松坂 和夫,線型代数入門,岩波書店 \\
\cite{liner}は線形代数学の諸概念に関する幾何学的な解説はほとんどない.
しかしこの本はそのあたりの説明が豊富になされている.
\cite{liner}よりも分厚いが,その分肉厚で詳しい解説がなされている.
値段は張るがその分の価値は間違いなくあるだろう.
 \bibitem{vec} 戸田 盛和,ベクトル解析(理工系の数学入門コース 3),岩波書店 \\
ベクトル解析の本.本書の記述ではあまりに足りないのでこういうしっかりとした本を読むべきである.
物理学者の言う「厳密さは置いておいていい」というのはこの本くらいのことを言うのである.
 \bibitem{tyo} 長沼 伸一郎,物理数学の直観的方法-理工系で学ぶ数学「難所突破」の特効薬,講談社 \\
巷ではもてはやされている本.私自身は中身を見たことはない. 
 \bibitem{den} 砂川 重信,電磁気学-初めて学ぶ人のために \\
電磁気学の本としてはかなり難易度は下げてある本.
しかし,その分レベルは低いかも.
  \bibitem{denj} 砂川 重信,電磁気学(物理テキストシリーズ 4),岩波書店 \\
上の本と同じ著者の本.とりあえず買っておけば後悔しない本.それくらいの名著.
  \bibitem{denjiki} 砂川 重信,理論電磁気学,紀伊國屋書店 \\
さらに同著者による本.こちらはMaxwell\index[nidx]{Maxwell@Maxwell(マクスウェル)}
方程式から始めてすべてを導くという体裁をとっている.
難易度はそれなりに高いものの,その分内容も濃く,得るものも多い.
 \bibitem{aa} 丹羽雅昭,詳解 物理学の基礎,東京電機大学出版局 \\
物理学のいろいろな分野の基礎的な解説がなされた本.
論理性を重視しており,計算過程はかなり詳しく書いてある.
反面,式の解釈や物理量の意味などの解説は全然足りないので
計算のカンニング本として使うのがいいかもしれない.
 \bibitem{syumi} 広江 克彦,趣味で物理学,理工図書 \\
著者のこだわりが色濃く表れた本.院生が学部生に教えるような
ノリで書いてあり読みやすい.
しかし,論理展開が非常に早く,著者独自の見解が多いのでこの本だけで勉強するのはやめておくべき.
 \bibitem{latex} 奥村 晴彦・黒木 裕介,[改訂第7版] {\LaTeXe}美文書作成入門,技術評論社 \\
付録にも挙げた{\LaTeX}に関する参考書.たぶんこの本より優れた{\LaTeX}の解説書はないだろう.
TeX Liveをインストールできるインストローラーも付属しているが,
この本でインストールできるTeX LiveはTeX Live 2016であり,最新版であるTeX Live 2017ではないので
インストールするときはこの本ではなくウェブサイト経由でインストールすべきだろう.
 \bibitem{KETpic} CASTEX応用研究会,KETpicで楽々TEXグラフ,イーテキスト研究所 \\
 本書のグラフ描画に(一部だが)使ったのがこれ.かなり強力なのでぜひとも手を出すべき.
 最近はKETcindyなるものにパワーアップしたそうである.
\end{thebibliography}