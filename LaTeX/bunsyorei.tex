\begin{verbatim}
\documentclass[a4paper,12pt]{jsarticle} % ドキュメントクラス
% 「%」でコメントが入れられる %
%%% ここからプリアンブル %%%
\usepackage{amsmath,amssymb} % 数式用パッケージ
%%% ここまでがプリアンブル %%%
\begin{document} % 本文スタート
数列$\{ a_n \}$を,漸化式
\begin{align} % 式番号付きの別行立ての数式環境のスタート
 \begin{aligned} % 複数行の式をひとまとめにして扱う
  a_1 & = 1 \\ % \\は改行を表す
  a _ {n+1} & = \sqrt{ a_n + 6} % \sqrtは平方根
 \end{aligned}
\end{align} % 数式環境終わり
によって帰納的に定義する.

% 段落が変わるのよ %

このとき,
\begin{align} 
\forall \varepsilon >0 \, \exists N \in \mathbb{N} \, 
\forall n \in \mathbb{N} 
( n \geq N \rightarrow \lvert a_n -3 \rvert < \varepsilon )
\end{align} 
が成り立つので,この数列$\{ a_n \}$は3に収束する,つまり,
\begin{align}
\lim_{n \to \infty} a_n = 3
\end{align}
となる.
\end{document}
\end{verbatim}