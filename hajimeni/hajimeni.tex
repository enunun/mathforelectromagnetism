\chapter{はじめに}
電磁気学は初学者にとって難しい.よくそう言われる.
なぜかと聞いてみれば,返ってくるのは大概「あの$\partial$とかいう記号の意味が分からない」
だの「ベクトルの外積の意味が分からない」とかいうものである.
そう,誰も物理の方でつまずいておらず,それ以前の数学の話でつまずいているのである.
まったく嘆かわしい限りである.電磁気学を学んでいるつもりが,
それにともなう数学のせいで,電磁気学をまるで学べていないのである.

とはいえ,肝心の電磁気学が数学さえできていさえすれば簡単かといわれるともちろんそうではない.
何も考えず,ただ形式的に計算を実行するだけであれば特に苦労はしないだろう.
もちろんそんな学習には何の意味もない.場の概念,エーテル理論,電場や磁場の実在性,
Maxwell\index[nidx]{Maxwell@Maxwell(マクスウェル)}方程式のGalilei\index[nidx]{Galilei@Galilei(ガリレイ)}
変換等々,考えるべきことは山ほどあるのだ.

そして,電磁気学の先に待っているのが特殊相対論である.
特殊相対論は光速度に関する理論だと思われることは多いが,
あれはあくまでオマケで,慣性系同士の関係を考察した理論である.
特殊相対論の発端は電磁気学である.
電磁気学の中に生じた問題点を考察した結果生まれた理論が特殊相対論なのである.

電磁気学で使う数学的概念には,高校では扱わない事柄がたくさん出てくる.
偏微分やベクトル解析が主にそれである.微分や積分に関する考え方も改めなければならない.
物理数学ということで,大学の担当の先生がかなり雑な定義をしたり,
厳密なことはいいから解釈だということが多いと思う.
しかしそれでは何もわからない.
当の先生本人は豊富な経験からその解釈を導き出しているのだろうが
受け取る側はそうではない.
そんな話を最初に聞いても``なんとなくわかった''ようにしかならない.
学問において``なんとなくわかった''は``何もわかっていない''と同義である.

では,どうすればいいのだろう? もちろん明確な答えがあるわけではないが,
1つ気を付けてほしいことがある.
それは,自分が学習をするときに``何の話をしているのか''や
``何を計算しているのか'',``このことが全体の議論の中でどんな意味を持つのか''
をはっきりと認識するようにしておくことである.
このようなことは本文中に書いておくと邪魔になることが多く,
多くの専門書では章の先頭のイントロダクションや
前書き等に書かれていることが多い.
そのあたりを読まずにいきなり本文から読み始める人は多いのではないだろうか.
前書きやイントロダクションはその理論の雰囲気をつかむためにとても重要である.
これを学びながらつかんでいくのはとても難しい.
それをその分野において世界の最先端を突き進んでいる専門家が
読者のためにわざわざ書いてくれているのである.
これを読み飛ばす手はないだろう.
これは,特に数学書において顕著である.
よく数学書は無味乾燥で不親切だといわれることが多いが本当にそうだろうか? 自分の過去を振り返ってほしい.
前書きやイントロダクションをすっとばして定義や定理とその証明,
および問題と解答しか読んでいなかったのではないだろうか? そんな読み方をしておいて
``数学書は無味乾燥で不親切だ''なんて言ってもそれは自分の読み方のせいであるといわざるを得ない.

また,学習を進めるうえで何かわからないことがあれば,それを定式化し,
はっきりと何がわからないかを正確に書いてほしい.
そうすると,たいていの疑問は解決する.
``何がわからないか''がわかれば``何を知ればよいか''がわかるからである.
電磁気学に限らずどんな学問分野でも``なんとなくわかった''を許していては進歩はしない.
まず第一歩として``わかった''のか``わかっていない''のかをはっきりさせることから始めてほしい.
``わからない''のはもちろん悪いことではないが,``なんとなくわかった''
は駆逐すべき悪である.
``よくわからないが,それは置いておいてとりあえず先に進む''というのも学習するうえでは非常に重要である.
しかしそのようなときもなにを置いておいたのかをはっきりさせてほしい.
そうしておけば,先の場面でそれが解決されたときに真っ先に気づくことができるだろう.

本書はそういう学ぶ以前のものを解決することを目的として書いた本である.
本書を読んだ後,電磁気学の本を再度読んでほしい.
驚くほど早く読めるはずだ.
自分がいままでつまずいていたことはこんなくだらないことだったのかと思ってもらえたら何よりである.

なお,本書は高校レベルの数学は一応既習ということにさせてもらった.
また,現在高校で扱わない事項のうち,線形代数学の行列計算と行列式のところまでの事項も
既習という前提で書いている.
とはいえ,そんなに深く学んでおく必要もないのでささっと確認する程度でいい.

このあたりの数学的概念はよく難しいといわれるが,
偏微分もベクトルの外積も線積分や面積分も,
きちんと順を追って考えれば高校生にだってわかる簡単な概念である.
最初は初見の記号だらけでビビるかもしれないが,きちんと最後まで読み通してほしい.
誰でも1回読めばわかる程度には詳しく書いたつもりである.
本書を読むことによって救われる人が1人でもいれば,この本を書いた甲斐もあったといえるだろう.

また,優れた組版システムとしての{\LaTeX}の紹介を付録として付け加えた.
読者の方々が文書を作成するのに真っ先に使うのはおそらくMicrosoft社のWordだと思う.
しかし,{\LaTeX}という組版システムを用いることで,もっときれいで美しい文書を簡単に作成することができるのである.
本書は{\LaTeX}を用いて執筆している.
組版に関して普通に専門書を読むだけではたぶん触れることはないだろうから,
本書の内容だけでなく組版に関しても見ていただけるとありがたい.
正直理系なら{\LaTeX}は一般常識といってもいい.
これを機に,専門分野の内容だけでなくそれを支えるテクノロジーにも
興味を持ってもらいたい.

\begin{flushright}
2017年 10月 著者 
\end{flushright}