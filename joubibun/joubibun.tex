\chapter{微分積分学の復習}
\section{常微分} %偏微分は別のファイルで%
\subsection{微分とは微小変化である}
微分について軽く復習しておこう.
といっても高校ではあまりやらないような話を中心にするので
すでに微分に習熟している人も読んでいってほしい.
この節の表題は常微分となっているが,今は特に気にしなくていい.

量と量との間に何らかの関係があるとする.片方の量が決まればもう片方の量が決まる.そんな関係である.
この関係について調べたい.それには片方の量とそれに対応する量がどうなっているかを調べるよりほかはないだろう.
ではどうすればいいだろうか? 一番わかりやすいのは量と量との対応関係を図示してしまうことである.
図を見ればこの``関係''がどのようなものか一目でわかる.これを\emph{グラフ}というのだった.
では,グラフはどのようにして書いたらいいだろうか.ここで登場するのが変化量という考え方である.
まず始点をどこか適当に決めておく.そして,片方の量をほんの少しだけ変化させる.
するともう片方の量もそれにしたがって変化するはずだ.
片方の量の変化がほんの少しであれば,もう片方の量の変化もほんの少しであるに違いない.
しかし,いくらほんの少しであるといっても,まったく同じであるとはいえないだろう.いくらかの差異があるはずだ.
ではどれほど違うのか.それが知りたい.知りたいのだ.
これがわかりさえすればグラフは簡単にかけてしまう! これこそ微分法の動機である.動機を知るのは大切である.
 動機を知れば自分がいまどこにいて,何をすべきかがはっきりするからだ.

さて,ここでは``量''として``数''を,``関係''として``関数''を考えることにする.
変化量を考えるには足し算(引き算)ができるものでないといけない.掛け算(割り算)もできればなおよしである.
そんな中で扱いやすいのが数というわけである.もっといえば実数や複素数である.
これ以外の数はちょっと扱いにくいので本書では触れないでおく.もっとも,ここで扱うのは実数のみである.
実数と実数の関係の中で最もポピュラーなのは関数だろう.
ある数を決めたときにそれに対応して別の数がただひとつ決まるとき,この対応関係のことを
\emph{関数}\index[widx]{かんすう@関数}というのだった.
このただひとつというのがミソである.ひとつの数にたくさんの数が対応してしまったら,
いったいどれを追いかけたらいいのかわかったものではない.
人間の脳は一度に複数のものを追いかけられるようにはできていないのである.

もちろん,これ以外の量や関係も考えられる.そのうちベクトルに関しては本書で扱う.
難しく見えてしまうかもしれないが,実数のときの考え方の多くを流用できるので実はあまり難しくない.
だからとにかく今は実数と関数について考えていくことにする. 

本題に入るとしよう.$x$の関数$f(x)$を考える.$x$が$x$から$x + \varDelta x$に変化したとき,$f(x)$のほうは$f(x)$
から$f(x+ \varDelta x)$に変化する.このとき,$f(x)$の変化量は$f(x+\varDelta x )-f(x)$である.
$f(x)$の変化なのでこれは$\varDelta f$と書ける.
そして,$x$の変化$\varDelta x$に対する$f$の変化$\varDelta f$の比率
\begin{align}
\frac{\varDelta f}{\varDelta x} = \frac{f(x+\varDelta x)-f(x)}{\varDelta x}
\label{eq:heikinhenkaritu}
\end{align}
を関数$f(x)$の区間$[x, \, x+\varDelta x]$における\index[widx]{へいきんへんかりつ@平均変化率}
\emph{平均変化率}というのだった.
$\varDelta x$というある程度広い幅を考えているので``平均''である.
そうそう,この``$\varDelta$''というのはギリシャ文字であり,``デルタ''と読む.
$\varDelta$はよく変化量を表すのに使われる.$\varDelta x$と書けばこれは$x$の変化を表すし,$\varDelta f$と書けばこれは$f$の変化を表す.
つまり何が言いたいかというと,$\varDelta$というのは単独の数ではなく,後ろの文字とくっついてその数の変化量を表すということである.
だから$\varDelta$を1つの数のように思って
$$
\frac{\bcancel{\varDelta}f}{\bcancel{\varDelta}x} = \frac{f}{x}
$$
のようにしてくれるなよということである.

さてと,平均変化率の式(\ref{eq:heikinhenkaritu})
の$\varDelta x$を限りなく0に近づけたときの極限を考える.この極限が$\varDelta x$の正負にかかわらず一定の値に収束するとき,
関数$f(x)$は$x$で\emph{微分可能}である\index[widx]{びぶん@微分!かのう@---可能}といって,
その極限値を
\begin{align*}
\frac{\mathrm{d}f}{\mathrm{d}x} \; , \; f'(x)
\end{align*}
などと書き,これを関数$f(x)$の\emph{導関数}\index[widx]{どうかんすう@導関数}というのだった.
\begin{align}
\frac{\mathrm{d}f}{\mathrm{d}x} = \lim_{\varDelta x \to 0} \frac{\varDelta f}{\varDelta x} = 
\lim_{\varDelta x \to 0}\frac{f(x+\varDelta x)-f(x)}{\varDelta x}
\label{eq:bibunteigi}
\end{align}
である.
物理学では時間が変化していくにつれて物理量がどのように変化していくかを考えることが多い.
たとえば位置$x$が時間$t$によって決まる関数$x(t)$で表されるとき,その時間微分(すなわち速度のことである)を
$\dot{x}(t)$のように書くこともある.これはNewtonが使っていた記法である.
時間の関数の上にドットがついていたら,それはその関数の時間微分を表すということである.
この記法は本書ではあまり使わないのでそれほど気にしなくてもよいだろう.
ただし,力学では非常によく使われるので注意しておこう.
さて,導関数の読み方であるが,``$\mathrm{d}$''を使った方は
``ディー・エフ・ディー・エックス'',$f'(x)$の方は``エフ・ダッシュ・エックス''
だの``エフ・プライム・エックス''などと読む.ドットの方は``エックス・ドット・ティー''などと読む.
``$\mathrm{d}$``を使った方でなぜこんなめんどくさい読み順をするのかというと,これが
ひとつの関数を表していて,決して{$\mathrm{d} x$}と{$\mathrm{d} f$}の割り算などではないと強調するためである.
だが本当にそうかと言われれば微妙なところである.詳細は後述することにする.

関数の導関数というのは$x$のみの関数であって,各点$x$での$f(x)$の変化率を表す関数である.
もともと区間$[x, \, x+\varDelta x]$における平均変化率を表していたものの$\varDelta x$を0に近づけたのだから,これが1点$x$での変化率を
表しているというのは当然だろう? このことから,導関数に各点の座標を当てはめたものはその点における
\index[widx]{しゅんかんのへんかりつ@瞬間の変化率}\emph{瞬間の変化率}と呼ばれることがある.
また,\emph{微分係数}\index[widx]{びぶん@微分!けいすう@---係数}とも呼ばれている.
そして,関数の導関数を求めることをその関数を\emph{微分する}\index[widx]{びぶん@微分!する@---する}
\footnote{``微分''ではなく``微分する''である.詳細は後述.}というのである.
\subsubsection{微分は接線の傾き?}
関数のある点における微分係数は,
その点における関数のグラフの接線の傾きを表す
というのは高校の教科書によく載っている話である.
読者の方にも接線の方程式を求めてうんぬんという問題を解いたことがある人は多いはずである.
だが,論理的にはむしろ逆であり,関数が微分可能であるような点$a$での微分係数$f'(a)$を用いて,方程式
\begin{align}
y-f(a) = f'(a)(x-a)
\label{eq:sessen}
\end{align}
の表す直線を関数$y=f(x)$の点$a$における接線と\.定\.義\.す\.るのである.
そしてこれは,微分法のよくある応用例のほんの一部分でしかない.
大学入試では接線に関する問題が頻出なので,これが微分法のすべてだと思い込んでしまいがちである.
しかし,物理学で接線を求めて何かする,というのはほとんどない.導関数とは関数の瞬間の変化率を表す関数なのである.
と書いてみたものの,手持ちの数学I\hspace{-,1em}Iの教科書を確認してみたところ,上と同じような文章が書いてあった.
どうやら高校の教科書が接線の扱いを間違っているというのは私の誤解だったようである.
\subsubsection{導関数と微小変化としての微分}
さて,ここからが本番である.もう一度導関数の定義式を見てみよう.
\begin{align}
\frac{\mathrm{d}f}{\mathrm{d}x} = \lim_{\varDelta x \to 0} \frac{\varDelta f}{\varDelta x}
\label{eq:doukansu}
\end{align}
この式は$\varDelta x$が無限に小さいという想定である.
しかし,無限に小さいまでとはいかなくとも,$\varDelta x$がある程度小さければ,
式(\ref{eq:doukansu})は成立するのではないのだろうか.
そうそう,ここでいう``小さい''というのは0に近いという意味である.
小さいといえば普通0に近いということを表す.無限に小さいだとか無限小というのは限りなく0に近いという意味である.
$\varDelta x$が無限に小さいというわけでなければ式の成立はあくまで近似的にとなる.
\begin{align}
\frac{\mathrm{d}f}{\mathrm{d}x} & \approx \frac{\varDelta f}{\varDelta x} \nonumber \\
\therefore \; \varDelta f & \approx \frac{\mathrm{d}f}{\mathrm{d}x} \varDelta x
\label{eq:henkaryou}
\end{align}
式(\ref{eq:henkaryou})の精度は$\varDelta x$が小さいほどよくなっていく.
当然だが$\varDelta x$が無限に小さければこの近似は驚くほど正確になり,
$\approx$\footnote{ほぼ等しいというのを表すのに日本ではよく$\fallingdotseq$という記号が用いられるが,
$\approx$や$\simeq$が国際標準である.読み方は全部``ニアリーイコール''である.}
ではなく$=$で結んでよいほどになる.
このようなとき,$\varDelta x$が無限小の変化量であることを強調するために$\varDelta x$の代わりに$\mathrm{d}x$という記号を使う.
もちろん$f$の変化も無限小であり,$\mathrm{d}f$という記号を用いて式(\ref{eq:henkaryou})は
\begin{align}
\mathrm{d} f = \frac{\mathrm{d}f}{\mathrm{d}x} \mathrm{d}x
\label{eq:bibun}
\end{align}
と書き換えられることになる.まるで微小な変化量どうしの割り算を実行したかのようである.式(\ref{eq:bibun})で用いられた
微小な変化量$\mathrm{d}x$や$\mathrm{d}f$を\emph{微分}\index[widx]{びぶん@微分}と呼ぶ.
変化量を偉そうな言い方をして\emph{変分}\index[widx]{へんぶん@変分}と呼ぶことがある.
微小な変分,略して微分である.そして,この微分どうしの比率を求めることこそが微分する,というわけである.
先ほど微分係数という言葉を使ったが,この言葉には微分どうしを結びつける係数という意味が込められていたということだ.
言葉の定義というのは実によくできているのである.とはいうものの,``微分''と``微分する''の2つの用語を厳密に区別するのは
面倒だしあまりいいこともないので一緒くたにして使ってしまうことも多い.私もさっき時間微分という言葉をしれっと使ったのであった.

ところで,私は前に微分するというのは微分どうしの割り算をすることなどではないと言った.
しかし,式(\ref{eq:henkaryou})や式(\ref{eq:bibun})
はまるで微分するということが微分どうしの割り算であると言っているようである.この点に関しては今も論争が続いている.
確かに微分するという行為が微分どうしの割り算を表すとイメージしてはいけないわけではないのだが,そのまま先へ進むと
ときどき間違った結果を導いてしまうことがある.特にこれから学ぶ偏微分がいい例である.
だがそれと同時に,微分どうしの割り算とみなした方が数式のイメージがしやすいときもたくさんある.
物理学者は数学的な厳密さなど気にも留めないので,
微分するというのは微分どうしの割り算をすることであるとする立場の
人間がほとんどである.そして,それで間違えたらそこで立ち止まってじっくり考えたらいいじゃないかというスタンスである.

このような立場を支持する例を挙げておこう.$x$の関数$f(x)$における$x$が別の変数$t$の関数であったとする.
$t$が変化すれば$x$が変化し,それに応じて$f$も変化する.$f$の$t$に対する変化率を求めたい.

$t$が微小量$\mathrm{d}t$だけ変化したとき,$x$は
$$
\mathrm{d}x = \frac{\mathrm{d}x}{\mathrm{d}t} \mathrm{d}t
$$
だけ変化するのであった.これを式(\ref{eq:bibun})に代入して
$$
\mathrm{d}f = \frac{\mathrm{d}f}{\mathrm{d}x}  \frac{\mathrm{d}x}{\mathrm{d}t} \mathrm{d}t
$$
となる.求めたいのは$\mathrm{d}f$と$\mathrm{d}t$の比であるので,両辺を$\mathrm{d}t$で割って,
\begin{align}
\frac{\mathrm{d}f}{\mathrm{d}t} = \frac{\mathrm{d}f}{\mathrm{d}x}  \frac{\mathrm{d}x}{\mathrm{d}t}
\label{eq:rensaritu}
\end{align}
と,高校の教科書でよく見る合成関数の微分法の公式が導かれる.

こうしてみると,``微小量同士の割り算''というわかりやすいイメージがあるのに
なぜ微分があんなめんどくさい定義になっているのだろうか,という疑問が湧いてくる.
その疑問に対する答えは至極簡単であり,
微小量同士の割り算といってもいったい何を求めるのかよくわからないからである.
例として,関数$f(x)=x^2$を$x$で微分することを考えよう.
$x$が微小量$\mathrm{d} x$だけ変化したとき,$f$の変化$\mathrm{d} f$は
\begin{align*}
\mathrm{d} f = ( x + \mathrm{d} x ) ^2 - x^2 = 2x \, \mathrm{d} x + ( \mathrm{d} x ) ^2
\end{align*}
従って,$\mathrm{d} f$と$\mathrm{d} x$の比は
\begin{align*}
\frac{ \mathrm{d} f } { \mathrm{d} x } = 2x + \mathrm{d} x
\end{align*}
ということになる.
しかし我々は
\begin{align*}
\frac{ \mathrm{d} f } { \mathrm{d} x } = 2x
\end{align*}
であることを``知っている''はずである.$\mathrm{d} x$がついているかどうかという差が出た.
こういうときによく``$ \mathrm{d} x $は$2x$に比べて非常に小さいので無視する''とか
$\mathrm{d} f$を求めるときに``$( \mathrm{d} x ) ^2$が
$\mathrm{d} x $に比べて非常に小さいから無視することにして''
とかいう理屈で計算が進められる.
ところが,$\mathrm{d} x$はいくら小さいといっても0とは言えないはずだ.
ぴったり0ではないが0に限りなく近いというのがミソであった.
後でやるが,1次近似という言葉を聞いたことがある人は
$\mathrm{d} x$は残して$( \mathrm{d} x )^2$
は消すということについては当然だと思うかもしれない. 
しかしこんな例もある.$f(x) = \sin x$としてみると,
$x$が微小量$\mathrm{d}x$だけ変化したときの$f$の変化量$\mathrm{d}f$は
\begin{align*}
\mathrm{d} f & = \sin (x + \mathrm{d} x ) - \sin x \\
 & = \sin x \cos \mathrm{d} x + \cos x \sin \mathrm{d} x - \sin x
\end{align*}
 となり,従って$\mathrm{d} x$と$\mathrm{d} f$の比は
\begin{align*}
 \frac{ \mathrm{d} f } { \mathrm{d} x } = \frac{ \sin x \cos \mathrm{d} x }{\mathrm{d} x} 
 + \frac { \cos x \sin \mathrm{d} x } { \mathrm{d}x } - \frac { \sin x } { \mathrm{d} x }
\end{align*}
となって,これ以上計算できない. 
賢い人はここで``$\cos \mathrm{d} x$は$\mathrm{d} x$が限りなく小さいので1とみなしてよくて,
$\sin \mathrm{d} x$の方も同じ理由で$\mathrm{d} x$
とみなせるから結果は$\cos x$になる''とかいう理屈を持ち出す.
だが,その近似は本当に正しいだろうか? 実を言うと,
この近似は三角関数の微分をもとに正当化されるのである.
従って,この論法は循環論法ということになる.
\footnote{まさかここでお絵かきをして,この図からこういう近似式が成り立っていて…
なんて言い出す愚か者はいないだろうとは思っている.}

ここまでで何が言いたいかといえば,
微小量同士の割り算というのは微分という操作の1つの解釈であり,
これを定義とすることはできないということである.
定義というのは曖昧であってはならない.
誰が計算しても同じ結果にならなければならないのである.
``微小量同士の割り算''というのは微分の定義とはいえないのである.
\footnote{数学的にいえば,この定義はwell-definedでないということである.}
物理学においては式や概念の具体的な解釈が重要となるが,
だからこそ何を定義とするかをしっかりと認識しておかねばならないのである.

では最初に書いた定義はきちんとしたものだろうか? 極限というものには``限りなく近づく''
という言葉が含まれていた.非常に曖昧である.
``限りなく''とはいったいどのくらいなのだろうか? 差が0.0001
ならいいのか? それとも0.00000001か? いや,
これでも不十分なのである.
そこで,この``限りなく近づく''という部分を明確化し,極限を厳密に議論した理論が
\textbf{$\varepsilon-\delta$論法}である.
よく難しいといわれる理論であるが,それは単に先入観によるものである.
きちっとていねいに考えていけば決して難しい理論ではない.
この本が読めるくらいのレベルであれば問題なく挑んでいい理論である.
\subsection{高階導関数}\index[widx]{こうかいどうかんすう@高階導関数}
さて,関数$f(x)$の導関数というのは$x$の関数であった.ならばこれの導関数も考えられる.
関数の導関数をさらに微分して得られる関数を\emph{$2$階導関数}
\index[widx]{にかいどうかんすう@2階導関数|see{高階導関数}},
\emph{第$2$次導関数}\index[widx]{だいにじどうかんすう@第2次導関数|see{高階導関数}}などといい,
$$
\frac{\mathrm{d}^2 f}{\mathrm{d} x^2} \; , \; f''(x)
$$
などと書く.読み方は``d''を用いた方は``ディー・ツー・エフ・ディー・エックス・ツー''で,
もうひとつの方は``エフ・ダブル・プライム・エックス''や
``エフ・ツー・ダッシュ・エックス''と読む.
\footnote{$'$をダッシュと読む人はツーと読むことが多く,プライムと読む人はダブルと読むことが多いそうだ.
ちなみに私はツー・プライム派である.}
``d''を用いた記法のところの2の位置に気を付けよう.これは,
$$
\frac{\mathrm{d}}{\mathrm{d}x} \left( \frac{\mathrm{d} f}{\mathrm{d} x} \right) = \frac{\mathrm{d}^2 f}{\mathrm{d} x^2}
$$
というところからきている.

まったく同様にして\emph{$n$階導関数}
\index{えぬかいどうかんすう@$n$階導関数|see{高階導関数}}というのも考えられる.
これは,関数を$n$回微分して得られる関数であり,
$$
\frac{\mathrm{d}^n f}{\mathrm{d} x^n} \; , \; f^{(n)} (x)
$$
などと書く.プライムを$n$個書くわけにもいかないのでかっこで微分した回数をメモしているのである.
また,2階,3階導関数などは導関数という言葉を使わずに2階微分,3階微分などと言ってしまうこともある.
漢字の使い分けに注意してほしい.名詞的に使うときは2階微分,3階微分という風にいうが,
動詞的に使うときには2回微分する,3回微分するというようにいう.

次は微分公式の話をしたいのだが,その前に関数の和や実数倍について話しておこう.

数$x$に対し数$f(x)$を対応させるような関数$f$と,数$x$に対し数$g(x)$を対応させるような関数$g$を考える.
この2つの関数に対し,新しい関数$f+g$を,$x$に対し$f(x)+g(x)$を対応させるような関数と定義する.
つまり,
\begin{align}
(f+g)(x) = f(x)+g(x)
\label{eq:kannsuuwa}
\end{align}
である.同じように関数の差や実数倍や積,商,合成も定義する.つまり,
\begin{align}
(f-g)(x) & = f(x) - g(x)
\label{eq:kannsuusa} \\
(cf)(x) & = cf(x)
\label{eq:kannsuujisuubai} \\
(fg)(x) & = f(x)g(x)
\label{eq:kannsuuseki} \\
\left( \frac{f}{g} \right) (x) & = \frac{f(x)}{g(x)} 
\label{eq:kannsuusyou} \\
( g \circ f) (x) & = g( f(x))
\label{eq:kansuugousei}
\end{align}
ということである.関数の合成だけはちょっと別だが,これらの式の右辺は実数の普通の演算である.
しかし,左辺は関数の和,差,実数倍,積,商というまったく別の演算であることに注意しなくてはならない.
なぜこのような演算を考えるかはここで話すより線形代数学を学んだ方が早いだろう.
いずれにせよ,物理学を学ぶ上では必須科目である.

ここでやっと微分公式が登場する.微分可能な関数$f, \, g$に対して,
その和や実数倍などの関数も微分可能であって,次の式が成り立つ.
\begin{align}
(af + bg)'(x) & = af'(x) + bg'(x)
\label{eq:bibunnsennkeisei} \\
(fg)'(x) & = f'(x)g(x) + f(x)g'(x) 
\label{eq:sekibibunn} \\
\left( \frac{f}{g} \right)'(x) & = \frac{f'(x)g(x) - f(x)g'(x)}{g(x)^2}
\label{eq:syoubibunn} \\
(g \circ f )' (x) & = g' \left( f(x) \right) f'(x)
\label{eq:gouseikannsuubibunn}
\end{align}
上の式の中でも特に重要なのは式(\ref{eq:bibunnsennkeisei})である.
この式は,微分という操作が線形性という性質を持っていることを示している.詳細は線形代数学を学んでもらいたい.

一例として,式(\ref{eq:sekibibunn})を導出してみよう.
微分可能な関数$f(x), \, g(x)$に対し,$h(x)=(fg)(x)$とおく.
$h$の微分$\mathrm{d}h$は
\begin{align*}
\mathrm{d}h = \frac{\mathrm{d} h} { \mathrm{d} x } \mathrm{d} x
\end{align*}
であるが,この$\mathrm{d}h$というのはそもそも$h(x)$の変化量のことであったから,
\begin{align*}
\mathrm{d} h & = h(x + \mathrm{d} x) - h(x) \\
& = (fg)(x+\mathrm{d}x) - (fg)(x) \\
& = f(x + \mathrm{d} x ) g (x + \mathrm{d} x) - f(x)g(x) \\
& = \left( f(x) + \frac{ \mathrm{d} f } {\mathrm{d} x } \mathrm{d}x \right)
\left( g(x) + \frac{ \mathrm{d} g } {\mathrm{d} x } \mathrm{d}x \right) - f(x)g(x) \\
& = f(x)g(x) + f(x) \frac{ \mathrm{d} g } {\mathrm{d} x } \mathrm{d}x
+ \frac{ \mathrm{d} f } {\mathrm{d} x } g(x) \mathrm{d}x - f(x) g(x) 
+ \frac{ \mathrm{d} f } {\mathrm{d} x } \frac{ \mathrm{d} g } {\mathrm{d} x } ( \mathrm{d} x ) ^2 \\
& = \left( f'(x) g(x) + f(x) g'(x) \right) \mathrm{d}x
 + f'(x) g'(x) ( \mathrm{d} x ) ^2
\end{align*}
 というように計算できるが,$( \mathrm{d} x )^2$というのは$\mathrm{d} x$よりもはるかに小さく,
 この項を無視することにすると,
 \begin{align*}
 \frac{ \mathrm{d}h } {\mathrm{d}x } = f'(x) g(x) + f(x) g'(x)
 \end{align*}
 であることが導かれる.
 それほど難しい計算でもないと思う.
 \begin{itembox}[l]{課題}
先ほど示した積の微分公式の導出にならい,残りの微分公式を導出せよ.
\end{itembox}
微分公式の導出については後でもう一度振り返ることにする.

そういえば,微分公式といえばこんなのもあるのであった.
\begin{align}
(x^\alpha)' & = \alpha \, x ^{\alpha-1} \\
(e^x)' & = e^x 
\label{eq:expbibunn} \\
( \sin x )' & = \cos x 
\label{eq:sinbibunn} \\
( \cos x )' & = - \sin x 
\label{eq:cosbibunn} 
\end{align}
特に重要なのが式(\ref{eq:expbibunn})である.指数関数は微分してもその関数形が変わらないのである.

さて,物理学では指数関数の指数の部分が
$e^{i(k_x x + k_y y + k_z z - \omega t)}$というように複雑になるということがよくある.
このようなときに指数を$e$の右上に小さく書くのは骨が折れる.
というわけで,指数関数を$\sin x$や$\cos x$のように指数(exponential)の上から3文字をとって 
$\exp (x)$と書き表すことがある.というより非常によく使う.

さっきの例だと$\exp (i(k_x x + k_y y + k_z z - \omega t))$
という風に書けて幾分かすっきりするし,手書きで書くときにはこちらの方が書きやすい.
なお,読み方は$e^x$は``イーのエックス乗''と読む人が多かったと思うが,
$\exp (x)$はexponentialをそのまま読んで``イクスポネンシャル・エックス''か
``エクスポネンシャル・エックス''と読むことの方が多いように感じる.
どちらの記法を使うかはその人の好みによるところが大きいのだが,いずれにせよ,どちらの記法が出てきたとしても
これは指数関数なんだなぁとわからなければならない.私は$\exp$で書く方が好きなので本書ではそちらで書くことにする.

微分法のイメージはつかめただろうか.本書で使わないような定理や公式は他の本に譲るとして,
次は応用上非常に重要なTaylor展開について学ぼう.

\section{{\rm Taylor}展開と{\rm Euler}の公式}
\subsubsection{近似の精度}
$x$の関数$f(x)$において,点$x$がある点$a$からほんの少しだけ離れた点であるとする.そして,点$a$での$f(x)$の値$f(a)$と
その微分係数$f'(a)$はわかっているものとする.
この情報を頼りに$f(x)$の点$a$の
近傍\footnote{``点$a$の近傍''とは点$a$の近くという意味である.正確な定義は数学書を見てほしい}
での振る舞いを調べたい.

それには,$f$の点$a$での変化率$f'(a)$がわかっているのだから,式(\ref{eq:henkaryou})をちょこっと変形して
\begin{align}
f(x) \approx f(a) + f'(a)(x-a)
\label{eq:ikkaikinnji}
\end{align}
としてやればよい.しかし,式(\ref{eq:ikkaikinnji})の成立はあくまで近似的にであって正確ではない.
近似には精度がつきものである.
関数$f$の形や$x$が$a$からどれだけ離れているかによってはこの式は使い物にならないほど精度が悪いものになってしまうかもしれない.
どうにかしてこの式の精度を上げたい.できるのならば完全な等号にしてしまいたい.
このための技法がTaylor展開である.

\subsubsection{近似はなぜ必要か}
そういえば,そもそもなぜ近似なんてものをするだろうか? 近似をすれば真の値からずれてしまう.近似を繰り返し
適用すると,誤差が膨れ上がってもはや役に立たない,というのもよくある話である.これは,測定機器の性能によって
生じるいわば現象論的な誤差とは全く別の問題である.

この問いに対する答えはいたって簡単なものである.近似を使うのは式が複雑すぎて
計算を先に進めることができないときなのである.要は計算がしんどかったりそもそも解けるかも怪しかったりしたときに,
厳密な解はとりあえず諦めてそれと近い近似解で満足しているのである.
どうせ測定機器の性能が原因で誤差は出るのである.今さら理論面で誤差が出たところでなんだというのだ.
別に不都合はないではないか.そんなイメージである.それに,近似式で生じる誤差ならばある程度正確に計算することができる.
これについての具体的な話はそういう本を読んでもらえばよいだろう.なかなかに奥が深い話である.

そして,近似を行うのは何も数式に限った話ではない.そもそも物理学の基本的な目的は自然界を理解することである.
だが,自然界は人間が理解するのは少々複雑すぎる.それは一見単純に見える要因が幾重にも重なることによって
そうなっているのだ.そのなかには結果に対してそれほど深く影響を与えないものも多い.だからそんなものは無視するのだ.
そして人間にも理解できるくらい単純な構造へと帰着させる.
このようにして,物事(とその仕組み)をその本質を崩さないように単純化することを
\emph{モデル化}\index[widx]{もでるか@モデル化}といい,モデル化によって出来上がったものを
\emph{モデル}\index[widx]{もでる@モデル}という.
モデル化して得られた結果は現実とさほど離れてはいないはずである.
もし現実離れした結果が出てしまったとしたら,
それはモデル化をする段階で無視してしまった要因の中に無視してはならない重要なものが紛れ込んでいたわけだ.
だからそれを加味してもう一度理論を作り直せばよい.物理学はこの繰り返しによって発展してきたのである.
よく自分のやったものが正しいか正しくないかでしかその価値を判断できない人がいるが,
そんな低次元の考え方で学問はするべきではない.何が正しいのかなんてのはあとになってみないとわからないのだ.
この本を手に取るレベルの読者の方々はとっくにそういうレベルにまで達している.自身を過小評価すべきではない.

それに,こんな理論もある.
ある系の振る舞いがすでに正確にわかっているとして, そこから状況をほんの少し変化させたとしたら,
いったい系の振る舞いはどの程度変化するのだろうかというものだ.
このような理論は摂動論と呼ばれている.代表的なものは天体の振る舞いである.
天体は,他の天体から重力による影響を受けており,その運動を解析するのは並大抵のことではない.
しかし,例えば太陽系の惑星が太陽からのみ影響を受け,他の惑星や天体からは影響を受けないとすれば話は別だ.
割と簡単に惑星の軌道を解析することができる.力学の本には必ず載っている内容である.
しかしそれは,惑星の正確な軌道を表せてはいない.他の天体からの影響を無視しているからだ.
だが,その影響は太陽からのものに比べれば大したものではない.そこで摂動論が活躍する.
他の天体からの影響が入ることで,太陽からのみ影響を受けるとした軌道からほんの少しだけずれるのである.
こうすることで得られる解もやはり近似的である.
もともと正確に解けるような問題でもないのだからそれで満足しておくことにしよう.
こうして,そのままでは太刀打ちできない問題がとりあえず解けたことになるのだ.詳しい話はその手の教科書に当たってほしい.
\subsection{近似式から\bf{Taylor}展開へ}
話を戻そう.もう一度式(\ref{eq:ikkaikinnji})を見てほしい.
左辺の$f(x)$の形に特に制限はない.ただ微分可能でありさえすればよい.
対して右辺はどうだろうか.
$f(a)$も$f'(a)$も$x$に$a$を代入しているのでただの定数であり,1次式である.
つまり式(\ref{eq:ikkaikinnji})は関数を1次式として近似するための式なのである.
このことから,式(\ref{eq:ikkaikinnji})は$f$の点$a$周りの
\emph{$1$次近似}\index[widx]{いちじきんじ@1次近似}と呼ばれている.
近似式といえばほとんどがこれである.
高校物理でよく出てくる$x$が十分小さいとき,
つまり0に近いときの近似式$\sin x \approx x$や$\sqrt{1+x} \approx 1+x/2$などもすべて1次近似である.
試しに導出してみるといい.それほど手間のかかる作業でもない.

…導出は終わっただろうか? 1次近似ときたら次は2次近似である.
そのためにも,1次近似の式を少し違うアプローチで導出してみよう.

まず,関数$f(x)$がある点$a$の周りで1次式で近似できたと仮定する.
\begin{align}
f(x) \approx a_0 + a_1 x
\label{eq:itiji}
\end{align}
これから次数を増やしていくので昇べきの順で書いてある.
この式は1点$a$においては厳密に成り立っていなければならない.
\begin{align*}
f(a) = a_0 + a_1 a 
\end{align*}
また,式(\ref{eq:itiji})の両辺を$x$で微分して$x=a$を代入する.これも厳密に成り立つべきである.
\begin{align*}
f'(a) = a_1
\end{align*}
これにより$a_0 = f(a)-af'(a), \, a_1 = f'(a)$であることがわかり,晴れて式(\ref{eq:ikkaikinnji})が導かれる.
察しのいい人は関数を近似するときに
$$
f(x) \approx a_0 + a_1 (x-a)
$$
とおいた方がいいんじゃないかと思ったかもしれない.まったくその通りで,この方が次数を増やしていったときに
計算の負担が一気に減るし,見通しもよい.2次近似からは計算がしんどいのでこの方針でいくことにする.

さて,1次近似の次は2次近似である.関数$f(x)$が点$a$の周りで2次式で近似できたとする.
\begin{align}
f(x) \approx a_0 + a_1 (x-a) + a_2 (x-a)^2
\label{eq:niji}
\end{align}
$a_0, a_1, \, a_2$の値はどうなるだろうか.未知数が増えたので情報を増やさねばならない.
$f(x)$の点$a$における2階微分係数$f''(a)$がわかっているものとしよう.
やることはさっきと同じである.まず式(\ref{eq:niji})に$x=a$を代入すると$a_0 =f(a)$であることがわかり,
次に式(\ref{eq:niji})の両辺を$x$で微分して$x=a$を代入すると$a_1=f'(a)$であることがわかる.
さらにもういちど$x$で微分してから$x=a$を代入すれば$a_2 = f''(a)/2$であることがわかる.

したがって,2次近似の式というのは
\begin{align}
f(x) \approx f(a) + f'(a)(x-a) + \frac{f''(a)}{2}(x-a)^2
\label{nikaikinnji}
\end{align}
だというわけだ.ちょうど1次近似の式に補正項$f''(a)(x-a)^2 /2$が追加された形になっている.
この項のおかげで近似式の精度が向上しているのだ.

まったく同様にして3次近似や4次近似なども考えられるが,一気に$n$次近似までいってしまおう.
導出のやり方はさっきと同じなのでぜひやっておいてもらいたい.間違えなければ次のような式が導けるはずだ.
\begin{align*}
f(x) \approx f(a) \, + \, & f'(a)(x-a) + \frac{f''(a)}{2}(x-a)^2 \\
& + \frac{f'''(a)}{3!}(x-a)^3 + \cdots + \frac{f^{(n)}(a)}{n!}(x-a)^n
\label{nkaikinnji}
\end{align*}
$0!$も$1!$も1であるから,この式は$\sum$を使えばもっとすっきりまとめられる.
\begin{align}
f(x) \approx \sum_{k=0}^{n} \frac{f^{(k)}(a)}{k!} (x-a)^k
\end{align}
$n$が大きければ,この式の精度は1次近似の場合よりもずいぶんとよいはずである.
$n$が大きければ大きいほど精度はよいのである.ならばいっそ$n \to \infty$としてしまおうではないか.
式の精度は限りなく向上する.つまり近似ではなく完全な等式となる.
\begin{align}
\begin{aligned}
f(x) = & \sum_{n=0}^{\infty} \frac{f^{(n)}(a)}{n!} (x-a)^n \\
= & f(a) \, + \,  f'(a)(x-a) + \frac{f''(a)}{2}(x-a)^2 + \cdots \\
& \hspace{1.2cm}+ \frac{f^{(n)}(a)}{n!}(x-a)^n + \cdots 
\label{eq:talortennkai}
\end{aligned}
\end{align}
これこそが欲しかった式である.この式と1次近似以外の話はもう忘れてしまって構わない.

与えられた関数$f(x)$を式(\ref{eq:talortennkai})の右辺のように書き表すことを$f(x)$の点$a$周りの
\textbf{Taylor展開}\index[nidx]{Taylor@Taylor(テイラー)}\index[widx]{Taylorてんかい@Taylor展開}と呼ぶ.
これは点$a$周りの展開であるが,精度を気にしなくてよくなった以上,$a$の周りである必要はまったくない.
そこで,式が簡単になるように$a=0$として,
\begin{align}
\begin{aligned}
f(x) = & \sum_{n=0}^{\infty} \frac{f^{(n)}(0)}{n!} x^n \\
= & f(0) \, + \,  f'(0)x + \frac{f''(0)}{2}x^2 + \cdots + \frac{f^{(n)}(0)}{n!}x^n + \cdots 
\label{eq:maclaurintennkai}
\end{aligned}
\end{align}
とできる.これを特別に$f(x)$の\textbf{Maclaurin展開}\index[widx]{Maclaurinてんかい@Maclaurin展開}
\index[nidx]{Maclaurin@Maclaurin(マクローリン)}と呼ぶ.
式がきれいなので実用上はほとんどこれが使われる.

さて,これまで私はあたかも項を増やすごとに近似式の精度が向上していくかのような言い方をしてきた.
だがそれは真っ赤な嘘…とまでは言えないが正しいとは言えない.$f(x)$の形と$x$の値によっては項を増やすごとに
式の値が増大していって,ついには無限大に発散してしまうことだって考えられる.無限級数ではそういうことがたびたび起こるのであった.
こうなると項を増やすことがかえって式の精度を悪くしてしまうことになってしまう.
また,ここには示さないが級数が収束してもそれがもとの関数と一致しない例もある.
展開ができる$x$の(絶対)値の限界のことをその級数の\emph{収束半径}
\index[widx]{しゅうそくはんけい@収束半径}と呼ぶ.
収束半径がわかったら,それより小さい$x$では問題なく展開できる.
収束半径ちょうどのところでは収束することもあれば発散することもある.これは展開する関数によってさまざまである.
また,どんな$x$でも問題なく展開できることも多く,その場合は``収束半径は無限大である''と表現する.
本書で主に扱う指数関数と三角関数の収束半径は無限大である.

\subsubsection{近似は本当にできるのか}
さっきやったのは,Taylor展開を近似式をもとに導出することだった.
関数$f(x)$を1次式,2次式,…と近似していって,最終的に``無限次''にすることで等号を成り立たせたのであった.
しかし,この議論には大きな落とし穴がある.それは,
任意の関数が多項式関数を用いて近似できるという保証がまったくなされていないということである.
これまでの議論は``任意の関数が多項式として近似できるのであれば''有効な議論である.

もちろん,グラフを描いて図形的に考察することもできなくはないのだが,
いささか説得力に欠ける.
それに,グラフで見ても近似式の精度が計算できない.
前者はともかく後者は割と深刻な問題である.

話を戻そう.近似式が使える根拠として,以下の定理がある.
\begin{itembox}[l]{定理}
関数$f(x)$が区間$I$で$n+1$回微分可能であるとする.
このとき,区間$I$内の定点$a$と任意の点$x$に対し,
\begin{align*}
f(x) = f(a) +  f'(a) (x-a)  + \frac{f''(a)}{2!}(x-a)^2 & \\ 
+ \cdots 
+ \frac{f^{(n)}(a)}{n!}(x-a)^n & + \frac{f^{(n+1)}(c)}{(n+1)!}(x-a)^{n+1}
  \label{eq:taylorth}
\end{align*}
をみたす実数$c$が$a$と$x$の間に存在する.
\end{itembox}
この定理は\textbf{Taylorの定理}\index[widx]{Taylorのていり@Taylorの定理}と呼ばれている.
注目すべきは,右辺が多項式関数っぽい関数\footnote{$c$が$x$に依存するので右辺は多項式関数ではない}
であるにもかかわらず等号が成立している点と,
一番最後の$f^{(n+1)}$の部分に代入されているのが$a$ではなく$c$であるという点である.
つまり,Taylorの定理が主張しているのは,
$f^{(n+1)}$の部分に代入する値をうまく調節すれば完全な等号を成り立たせることができて,
代入する値も$a$と$x$の間に確実に見つかるということである.
$n$次近似(もしくは$n$+1次近似)の式と非常によく似た形であるが,少し違う.
どこが違うかは言うまでもないだろう.
この定理は存在定理であって,式中の$c$がいくつであるかは教えてくれない.
複数存在する可能性だってある.
ただし,$c$が見つかる範囲はわかるのである.
すると,こんなことができる.$c$としてありうるすべての値(ここでは$a$と$x$の間のすべての値)を考えてやり,
最後の項である
\begin{align*}
\frac{f^{(n+1)}(c)}{(n+1)!}(x-a)^{n+1}
\end{align*}
の最大値を求めてやる.そうすると,この最大値は,$f(x)$の$n$次近似式との差の最大値を表していることになる.
すなわち,この最大値は,{$f(x)$}の{$n$}次近似式の誤差の最大値を表すと考えられるのである.
$c$の値によっては実際の誤差はこれよりも小さいかもしれない.
しかし,この最大値よりも実際の誤差が大きくなることはありえない.最大値とはそういうものであった.
この項は使えそうである.よって,
\begin{align}
R_n = \frac{f^{(n+1)}(c)}{(n+1)!}(x-a)^{n+1}
\end{align}
とおいて,これを$f(x)$の$n$次の\emph{剰余項}\index[widx]{じょうよこう@剰余項}と名付けよう.
$R_n$は$n$次近似式の精度を表していることになる.
この$R_n$の(最大)値が十分小さければ,
\footnote{``十分小さい''がどの程度小さいことを意味しているかはその場面によってさまざまである}
$R_n$は無視しても問題ないことになり,
晴れて$n$次近似の式が使えるということになるのだ.
$n=1$の場合が1次近似である.
$R_1$が十分小さければ1次近似は問題なく使っていいのである.
そして,$n$が限りなく大きくなるとき,
すなわち$n \to \infty$のとき,$R_n$の値が限りなく小さくなるとしたらどうだろう.
これは,
\begin{align*}
\lim_{n \to \infty} R_n = 0
\end{align*}
が成り立っているということである.
これが何を意味するかといえば,無限級数
\begin{align*}
 f(a) \, + \,  f'(a)(x-a) + \frac{f''(a)}{2}(x-a)^2 + \cdots + \frac{f^{(n)}(a)}{n!}(x-a)^n + \cdots
 \end{align*}
と$f(x)$の差が0であることを意味する.
つまり,剰余項{$R_n$}が0に収束すれば,{$f(x)$}はTalyor展開できるのである.
近似ができる根拠を調べていたが,Taylor展開が行える条件まで見つけてしまった.
剰余項は思ったより使えそうである.

\subsubsection{\bf{Maclaurin}展開の具体例}
さて,Maclaurin展開を実際に行ってみよう.

まずは指数関数からである.指数関数$\exp (x)$は微分しても関数の形が変わらない.
\begin{align*}
\frac{\mathrm{d}^n}{\mathrm{d}x^n} \exp(x) = \exp (x)
\end{align*}
であるから,
\begin{align*}
\frac{\mathrm{d}^n}{\mathrm{d}x^n} \exp(0) = \exp (0) = 1
\end{align*}
である.したがって,指数関数$\exp (x)$のMaclaurin展開は
\begin{align}
\begin{aligned}
\exp(x) = & \, 1 + x +\frac{1}{2!} x^2 + \cdots + \frac{1}{n!} x^n + \cdots \\
= & \sum_{n=0}^{\infty} \frac{1}{n!} x^n 
\label{eq:sisuukannsuu}
\end{aligned}
\end{align}
となる.きれいな形である.ぜひとも頭に叩き込んでおいてもらいたい.
指数関数のMaclaurin展開の収束半径は無限大である.ここには求め方は書いていないが.

次は三角関数である.まずは正弦関数$\sin x$からいこう.そのためには$n$階の導関数を求めなくてはならない.
1つの式でまとめるにはちょっとしたテクニックが必要だが,間違えなければ次のような式が導けるはずだ.
\begin{align*}
\frac{\mathrm{d}^n}{\mathrm{d}x^n} \sin x = \sin \left( x + \frac{n \pi}{2} \right)
\end{align*}
これにより,
\begin{align*}
\frac{\mathrm{d}^n}{\mathrm{d}x^n} \sin 0 = \sin \frac{n \pi}{2}
\end{align*}
となることがわかる.$n$が$n =$0,1,2,$\cdots$と増加していくにしたがって,
この値が$0, \, 1, \, 0, \, -1, \, 0, \, 1, \, \cdots$とループしていることに気づければMaclaurin展開はすぐにできて
\begin{align}
\begin{aligned}
\sin x = & \, x - \frac{1}{3!}x^3 + \frac{1}{5!}x^5 - \cdots +\frac{ (-1)^n}{(2n+1)!} x^{2n+1} + \cdots \\
= & \sum_{n=0}^{\infty} \frac{(-1)^n}{(2n+1)!} x^{2n+1}
\label{eq:seigenn}
\end{aligned}
\end{align}
と展開できる.収束半径は無限大である.

余弦関数$\cos x$も同様である.ほとんど同じなので結果だけ書かせてもらおう.
\begin{align}
\begin{aligned}
\cos x = & \, 1- \frac{1}{2!} x^2 + \frac{1}{4!} x^4 - \cdots + \frac{(-1)^n}{(2n)!} x^{2n} + \cdots \\
= & \sum_{n=0}^{\infty} \frac{(-1)^n}{(2n)!} x^{2n}
\label{eq:yogenn}
\end{aligned}
\end{align}
と,$\sin x$と似たような式になる.収束半径はさっきと同じで無限大である.違うのは$x$の偶数次の式か奇数次の式かくらいである.そういえば,$\sin x$は奇関数で,$\cos x$は偶関数なのであった.

\begin{itembox}[l]{問}
$a_0, \, a_1, \, \cdots , \, a_n$を定数として,関数$f(x)$を
\begin{align*}
f(x) = a_0 + a_1 x + a_2 x^2 + \cdots + a_n x^n
\end{align*}
と定める.このとき,$f(x)$の$x=0$周りでの1次近似は
\begin{align*}
f(x) \approx a_0 + a_1 x
\end{align*}
であることを示せ.
\end{itembox}

よく``$\lvert x \rvert$が非常に小さいので$x^2$は無視する''とかいう計算が行われることがあるが,
それは単に1次近似をやりますよと言っているだけなのである.
\subsection{\textrm{Euler}の公式}
$\sin x$と$\cos x$のMaclaurin展開はずいぶん似た形をしていて,
それぞれ$x$の偶数次と奇数次の式なのであった.
そしてこれらをうまく組み合わせれば指数関数になりそうな気がしなくもない.そう,できてしまうのである.
それには複素数の力を借りる必要がある.

複素数というのは$i^2=-1$という実数の範囲では決して成立しえない関係式をみたす
新しい数\emph{虚数単位}を用いて構成される数であるが,詳しく話すと長くなるので割愛することにする.
興味があればいい本はいくらでもあるのでそちらを参照していただきたい.

話を戻そう.まず,指数関数のMaclaurin展開の式を再確認する.
$$
\exp(x) = \, 1 + x +\frac{1}{2!} x^2 +\frac{1}{3!} x^3 + \frac{1}{4!} x^4 + \cdots
$$
やや天下り的なのだが,ここに複素数$iz$を代入する.
$$
\exp(iz) = \, 1 + iz +\frac{1}{2!} (iz)^2 +\frac{1}{3!} (iz)^3 + \frac{1}{4!} (iz)^4 + \cdots
$$
$i^2=-1$を利用して,$i$が残る方と残らない方とで分けてやる.
$$
\exp(iz) = \, \left(1- \frac{1}{2!} z^2 + \frac{1}{4!} z^4 - \cdots \right) + i 
\left( z - \frac{1}{3!}z^3 + \frac{1}{5!}z^5 - \cdots \right)
$$
$\sin x$と$\cos x$のMaclaurin展開を思い出せば,次のような式が導ける.
\begin{align}
\exp (iz) = \cos z + i \sin z
\label{eq:Euler}
\end{align}

非常に美しい式である.この式は任意の複素数$z$について成り立つ式であり,
\textbf{Eulerの公式}\index[widx]{Eulerのこうしき@Eulerの公式}
\index[nidx]{Euler@Euler(オイラー)}と呼ばれる重要な式である.そして,$z=\pi$とすれば,
\begin{align}
\exp (i \pi ) = -1
\label{eq:Eulertousiki}
\end{align}
という,非常に美しいEulerの等式\index[widx]{Eulerのとうしき@Eulerの等式}
が出来上がる.Napire数\index[nidx]{Napire@Napire(ネイピア)}$e$,虚数単位$i$,
円周率$\pi$,そして負の数$-1$といった数学の神秘性を象徴するような数が
たったひとつの等号で結ばれているのだ! 

また,
\begin{align*}
\exp (-iz) & = \exp (i(-z)) \\
& = \cos (-z) + i \sin (-z) \\
& = \cos z - i \sin z
\end{align*} 
だから,この式と元のEulerの公式とを組み合わせれば,
\begin{align}
\sin z & = \frac{\exp (iz) - \exp (-iz)}{2i} \\
\cos z & = \frac{\exp (iz) + \exp (-iz)}{2} 
\end{align}
という,これもまた美しい関係式が導かれる.三角関数は指数関数で表すことができるのである.

このような美しい式の数々に感動するのもいいが,Eulerの公式は単に美しいというだけではない.
その応用は多岐にわたる.その例を1つ紹介しよう.

まず,三角関数は波を表す,というのは聞いたことくらいはあるだろう.$y=\sin x$のグラフはいわゆる波の形をしている.
サインカーブというやつである.
ところが,三角関数は計算上扱うのが少し面倒である.微分するたびに$\sin, \, \cos, \, \cdots$という風に
関数形が変化してしまうのである.そして,その問題を解決してくれるのがEulerの公式というわけである.
指数関数は微分しても関数形が変化しない.その特性が計算に圧倒的簡便さをもたらす.
時には複雑な式をすっきりと表すのにも貢献してくれたりもする.
``波は複素数である''だの``指数関数は波を表す''といった文言はこのあたりの話が根拠になっているのである.
例えば$\psi (x, \, t) = A \sin (kx-\omega t)$という波は,
$\varphi (x, \, t) = A \exp (i(kx-\omega t))$という波に相当する.
指数関数と三角関数は複素数を仲立ちとして互いに表裏一体の関係にあるわけだ.
複素数は数学者のいわば遊びであり,現実には存在しない虚構の論理である,と言っている人をよく見かけるが,
そんな時代はとっくの昔に終わっているのである.
複素数はもはや,自然界を数学的に記述する上でかかせないものになっているのだ.
\begin{itembox}[l]{問}
三角関数の加法定理は複素数を用いると簡単に導くことができる.
Eulerの公式と指数法則
\begin{align}
e^{ i \alpha } \cdot e^{ i \beta } = e^{ i ( \alpha + \beta ) }
\end{align}
を用いて,三角関数の加法定理
\begin{align}
\sin ( \alpha + \beta ) = \sin \alpha \cos \beta + \cos \alpha \sin \beta \\
\cos ( \alpha + \beta ) = \cos \alpha \cos \beta - \sin \alpha \sin \beta 
\end{align}
を導出せよ.ただし,$\alpha , \, \beta$は実数とする.
\end{itembox}

さて,もうひとつだけ寄り道をしてみようか.
\section{微分可能性と1次近似}
微分可能性のところで導関数は微小量の割り算かどうかという話をした.
そして,``導関数は微小量の割り算である''という立場で積の微分公式の導出をしたのであった.
ではそうでない立場の人はどういう風に議論を進めているかというと,
次のような定理を出発点にしている.
\begin{itembox}[l]{定理}
関数$f(x)$が点$a$で微分可能であり,かつ$A = f'(a)$であるための必要十分条件は,
点$a$の周りで$f(x)$が
\[
\lim_{x \to a} \varepsilon (x) = \varepsilon (a) = 0
\]
をみたすような点$a$の周りで定義された関数$\varepsilon (x)$を用いて
\[
f(x) = f(a) + A \, (x-a) + (x-a) \, \varepsilon (x)
\]
と表されることである.
\end{itembox}

この定理をよく見てみると,後半の式がちょうど$f(x)$の点$a$周りでの1次近似を誤差込みで表したものになっている.
この場合の誤差は$(x-a) \, \varepsilon (x)$ということになる.
$\displaystyle \lim_{x \to a}\varepsilon (x) = \varepsilon (a) = 0$という条件は,
$x$が$a$に近づくときにこの誤差が$x-a$よりも早く$0$に近づくことを要求し,
\footnote{いわゆる2乗のオーダーとかいうやつである}
さらに$x$がちょうど$a$のときには誤差が0であることを要求している.
このような制約のもとに1次近似が使えることと,
$f(x)$が点$a$で微分可能であることが同値であることを主張しているのがこの定理である.

この定理を出発点にしている立場の人は,
誤差を$(x-a) \, \varepsilon(x)$という形できちんと評価している.
従って,これを無視する立場の人よりも安全に理論を組み立てているといえるだろう.

さて,この立場のもとで積の微分公式を証明してみよう.
示したいのは以下の定理である.
\begin{itembox}[l]{定理}
関数$f(x)$が点$a$で微分可能であり,かつ関数$g(x)$が点$a$で微分可能であるとする.
このとき,関数$(fg)(x)$は点$a$で微分可能であって,
\begin{align*}
(fg)' (a) = f'(a)g(a) + f(a) g'(a)
\end{align*}
が成り立つ.
\end{itembox}
この定理の証明はそれほど難しいわけではない.
$f(x)$が点$a$で微分可能であり,かつ関数$g(x)$が点$a$で微分可能であることから,
$A=f'(a)$,$B=g'(a)$とおくと,点$a$の周りで定義され,さらに
$\displaystyle \lim_{x \to a} \varepsilon_1(x)=\varepsilon_1(a)=0$
をみたす関数$\varepsilon_1(x)$と,点$a$の周りで定義され,さらに
$\displaystyle \lim_{x \to a} \varepsilon_2(a)=\varepsilon_2(a)=0$
をみたす関数$\varepsilon_2(x)$がそれぞれ存在して,
点$a$の周りで$f(x), \, g(x)$がそれぞれ
\begin{align*}
f(x) & = f(a) + A \, (x-a) + (x-a) \, \varepsilon_1(x) \\
g(x) & = g(a) + B \, (x-a) + (x-a) \, \varepsilon_2(x)
\end{align*}
と表せる.従って,関数$(fg)(x)$は点$a$の周りで
\begin{align*}
(fg)(x) &= f(x) g(x) \\
& =\Bigl(  f(a) + A \, (x-a) + (x-a) \, \varepsilon_1(x) \Bigr) \\
& \hspace{2.5cm} \times \Bigl( g(a) + B \, (x-a) + (x-a) \, \varepsilon_2(x) \Bigr) \\
& = f(a) g(a) + A \, g(a) (x-a) + f(a) B \, (x-a) 
+ f(a) (x-a) \, \varepsilon_2(x) \\ 
& \hspace{1cm} + AB \, (x-a)^2 + A \, (x-a)^2 \, \varepsilon_2(x) 
+ g(a) (x-a) \, \varepsilon_1(x) \\ 
& \hspace{2cm} + B \, (x-a) ^2 \, \varepsilon_1(x) 
+ (x-a) ^2 \varepsilon_1(x) \, \varepsilon_2(x) \\
& = (fg)(a) + \Bigl( A \, g(a) + f(a) B \Bigr) (x-a) \\
& \hspace{0.75cm} + (x-a) \Bigl( f(a) \, \varepsilon_2(x) + g(a) \, \varepsilon_1(x) \\
& \hspace{1.5cm} + (x-a) \bigl( AB+ A \, \varepsilon_2(x) + B \, \varepsilon_1(x) 
+ \varepsilon_1(x) \, \varepsilon_2(x) \bigr) \Bigr)
\end{align*}
と書き表される.

ここで,$C=A \, g(a) + f(a) B = f'(a)g(a) + f(a)g'(a)$とおき,
点$a$の周りで定義される関数$\varepsilon(x)$を
\begin{align*}
\varepsilon(x) & = f(a) \, \varepsilon_2(x) + g(a) \, \varepsilon_1(x) \\
& \hspace{1.2cm} + (x-a) \bigl( AB + A \, \varepsilon_2(x) + B \, \varepsilon_1(x) 
+ \varepsilon_1(x) \, \varepsilon_2(x) \bigr)
\end{align*}
と定めると,$\varepsilon(x)$は
\begin{align*}
\varepsilon(a) & = f(a) \, \varepsilon_2(a) + g(a) \, \varepsilon_1(a) \\
& \hspace{1.2cm} + (a-a) \bigl( AB + A \, \varepsilon_2(a) + B \, \varepsilon_1(a) 
+ \varepsilon_1(a) \, \varepsilon_2(a) \bigr) \\
& = f(a) \cdot 0 + g(a) \cdot 0 + 0 \cdot ( A \cdot 0 + B \cdot 0 + 0 \cdot 0 ) \\
& = 0, \\
\lim_{x \to a} \varepsilon(x) & = 
\lim_{x \to a} \Bigl( f(a) \, \varepsilon_2(x) + g(a) \, \varepsilon_1(x) \\
& \hspace{1.2cm} + (x-a) \bigl( AB + A \, \varepsilon_2(x) + B \, \varepsilon_1(x) 
+ \varepsilon_1(x) \, \varepsilon_2(x) \bigr) \Bigr) \\
& = f(a) \cdot 0 + g(a) \cdot 0 + (a-a) \bigl( AB + A \cdot 0 + B \cdot 0 + 0 \cdot 0 \bigr) \\
& = 0
\end{align*}
をみたし,さらに点$a$の周りで$(fg)(x)$は
\begin{align*}
(fg)(x) = (fg)(a) + C \, (x-a) + (x-a) \, \varepsilon(x)
\end{align*}
と表せる.従って$(fg)(x)$は点$a$で微分可能であり,
\begin{align*}
(fg)'(a) = C = f'(a)g(a) + f(a) g'(a)
\end{align*}
となる.$a$は任意だったので,結局導関数に関して
\begin{align*}
(fg)'(x) = f'(x)g(x) + f(x)g'(x)
\end{align*}
となる.

\pageref{eq:sekibibunn}ページにあった証明と比較してとても長い証明となった.
計算量が明らかに増えている.なぜこうなったかは明らかであり,
それは関数の1次式からの誤差を正確に計算したからである.
今回の証明における関数$(fg)(x)$の1次式からの誤差は$(x-a) \, \varepsilon(x)$
である.\pageref{eq:sekibibunn}ページのときに出てきて小さいからと切り捨てられたのは
$f'(x)g'(x)(\mathrm{d}x)^2$という項で,
これは計算中に出てきた$AB \, (x-a)^2$に相当する.
それ以外に出てきたものが以前の計算では無視されていたものであり,
今回はそれが本当に無視してよいかを検証したことになる.
とはいえ,得られたものは結局同じであり,骨折り損のくたびれ儲けといえなくもない.
\begin{itembox}[l]{問}
上に述べた微分公式の導出を参考にして,
\pageref{eq:sekibibunn}ページに残された微分公式の導出をせよ.
\end{itembox}

\subsubsection{数学者 vs 物理学者?}
ここでは微小量に対して2つの立場を紹介したが,
別に物理学者と数学者の間で立場が分かれ,争っているというわけではない.
\footnote{よく言われる``数学者は厳密性にうるさくて,物理学者は大雑把である''
というのも話を分かりやすく例えるための真っ赤な嘘である.}
両者の立場は個人の好みによるところが大きいし,
別に相反する立場であるというわけでもない.
それに,どちらかの立場が間違っていて,
どちらかの立場が正しいという問題でもない.
``無限小''や``無限大''といった概念が数学的に厳密でないというように
書かれることが多いが,それはまったく歴史的な背景によるものであり,
数学者はとっくに``無限小''や``無限大''という概念を厳密に定式化することに成功している.
\footnote{超準解析と呼ばれる理論である.}
立場が分かれるといわれることがあるのは,厳密であるかということよりも,
むしろ無限小や無限大という概念を教育においてどう扱うかということによるところが大きいのだろうと思う.
こればっかりは厳密に定式化することは難しい.

そのようなイデオロギー的問題に読者が犠牲にならないよう,
今回は2つの立場の両方から考えてみたのである.
