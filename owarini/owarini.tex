\chapter{おわりに}
ここまで読んでみてどうだったろう.あれだけ難しいと感じていたはずの偏微分もベクトル解析も,実にあっけなく終わってしまった.
余計なことをつらつらと書き連ねても100ページ程度にしかならない.その程度のことしかやっていないのである.

これでいよいよ電磁気学を本格的に学ぶことができる.
\ruby{Maxwell}{マクスウェル}方程式を見てみれば,もはや知っていることしか書いていないはずだ.
偏微分も$\nabla$も攻略し終わった今,電磁気学を見通しよくスムーズに学ぶことができるはずだ.
ここに書いてない事項が山ほど出てくるかもしれないが,順を追ってしっかり考えれば簡単に理解できてしまうはずである.

\ruby{Maxwell}{マクスウェル}方程式は電磁気学の法則の集大成ではあるが,ゴールではない.
そこから何がいえるかをめぐってさらに議論が発展していく.
立ち止まっている暇などない.どんどん先に進んでいこう.
